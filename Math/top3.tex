\documentclass{msjproc}
\usepackage{amsmath, amssymb, amsthm, amsfonts}
\usepackage[dvipdfmx]{graphicx}

\begin{document}
	
	\title{{\bf SECTION 2 MAPPING}}
	\author{Zodiac Caulfield}{}
	\maketitle
	
	\begin{description}
		\item[\underline{{\large \bf 謝辞}}]
			 {\bf
					$$
					前回の命題の証明に関して、
					\bigcap_{i \in I} f (\bigcap_{i \in I} A_i)
					\subset \bigcap_{i \in I} f (A_i)
					$$
					は,やっぱり自明でありました.お詫び申し上げます.
				}
	\end{description}

	\begin{description}
		\item[\underline{Prop 2.5.2}] $f : X \to Y$ を写像とする.
		\newline
			 {\bf 1}. $A \subset X$ と $B \subset Y$ について,
			$$f(A) \subset B \iff A \subset f^{-1}(B)$$
			 {\bf 2}. $f$ による image について,{\bf (1) - (3)} が成り立つ.
			\newline
				  {\bf (1)} $A \subset X$, $A \subset f^{-1}(f(A))$.
				\newline
				  {\bf (2)} $A, A' \subset X, A \subset A' \Rightarrow f(A') \subset f(A)$.
				\newline
				  {\bf (3)} $(A_i)_{i \in I} \in X, 
				f(\bigcup_{i \in I} A_i) = \bigcup_{i \in I} f(A_i), 
				f(\bigcap_{i \in I} A_i) \subset \bigcap_{i \in I} f(A_i)$.
			\newline
			 {\bf 3}. $f$ による inverse image について,{\bf (1) - (3)} が成り立つ.
			\newline
				  {\bf (1)} $B \subset Y, f(f^{-1}(B)) = f(X) \cap B, 
				f^{-1}(Y \setminus B) = X \setminus f^{-1}(B)$.
				\newline
				  {\bf (2)} $B, B' \subset Y, B \subset B' \Rightarrow 
				f^{-1}(B) \subset f^{-1}(B')$.
				\newline
				  {\bf (3)} $(B_i)_{i \in I} \in Y, 
				f^{-1}(\bigcup_{i \in I} B_i) = \bigcup_{i \in I} f^{-1}(B_i), 
				f^{-1}(\bigcap_{i \in I} B_i) = \bigcap_{i \in I} f^{-1}(B_i)$.
		\newline
		\item[\underline{Proof.}]
			 {\bf 1}. 
	\end{description}

	\begin{description}
		\item[\underline{Definition2.}]
		 A point p is called a cluster point of a directed 
		family provided every open set about p intersects each element F of the family.
		\newline
		 (p を含む任意の開集合が, どの点族の要素 F とも共通部分を持つとき, 
		p は有向点族の収積点という.)
	\end{description}

	\begin{description}
		\item[\underline{Definition3.}]
		 A directed family $\cal{F}$ converges to a point p 
		if and only if every open set about p contains some element of the family.
		\newline
		 (有向点族 $\cal{F}$ が点 p に収束することと,  p を含む任意の開集合が
		点族のある要素を含むことは同値.)
		
		\item[\underline{Proof.}]
		 (十分条件):
		有向点族の定義より, 有向点族の二つの要素の共通部分は, 
		有向点族の要素として含む. また, それが点 p に収束するので, 
		有向点族 $\cal{F}$ の要素 F を番号付けして表したときに, 
		$\epsilon - N$ 論法的に考えて点 p に収束することを考えると, 
		\newline\newline
		\centerline{$\forall \epsilon > 0, \exists N \in \mathbb{N} 
		\>\> s.t. \>\> \forall n \in \mathbb{N} \>\> n > N 
		\Rightarrow \|F_n - p\| < \epsilon$}
		\newline\newline
		と表せる. また, p の $\epsilon$ 近傍を考えてやると, これも p を含む開集合である. 
		先程の有向点族に関する収束の主張より, $F_n$ は p の $\epsilon$ 近傍に含まれる.
		\newline
		 (必要条件):
		p を含む任意の開集合が点族のある要素を含むとすると, 
		p の $\epsilon$ 近傍と共通部分を持つ点族の要素が存在することになる.
		このことから, 有向点族が p に収束するといえる.
		\newline よってこの主張は正しい.\qed
	\end{description}

	\begin{description}
		\item[\underline{Definition4.}]
		 If $\cal{E}$ and $\cal{F}$ are directed families, 
		then $\cal{E}$ is a (directed) underfamily of $\cal{F}$ provided each 
		element of $\cal{F}$ contains some element of $\cal{E}$.
		\newline
		 ($\cal{E}$ と $\cal{F}$ が有向点族で, $\cal{F}$ のどの要素も $\cal{E}$ 
		のある要素を含むとき, $\cal{E}$ は $\cal{F}$ の (directed) underfamily
		であるという.)
	\end{description}

	\begin{description}
		\item[\underline{Exercises I\hspace{-.1em}I\hspace{-.1em}I}]
			 
			\begin{enumerate}\renewcommand{\labelenumi}{\arabic{enumi}.}
				
				\item
				 A point p is a cluster point of a directed family $\cal{F}$ 
				provided some underfamily of $\cal{F}$ converges to p.
				\newline
				 ($\cal{F}$ のある underfamily が点 p に収束するという条件のもとで, 
				p は有向点族 $\cal{F}$ の収積点である.)
				\item[\underline{Proof.}]
				 $\cal{F}$ のある underfamily が点 p に収束すると仮定すると, 
				underfamily も 有向点族であるので, p を含む任意の開集合が点族の
				ある要素を含む. p を含む任意の開集合が, どの要素 F とも
				共通部分を持つような, ある点族を考えたとき, 
				p は有向点族 $\cal{F}$ の集積点である.  \qed

				\item
				 A topological space X is a Hausdorff space if and only if 
				each directed family of sets in X converges to at most one 
				point in X.
				\newline
				 (位相空間 X がハウスドルフであることと, X の集合のどの有向点族も
				X の高々一つの点に収束することは同値である.)
				\item[\underline{Proof.}]
				 (十分条件):
				ハウスドルフ空間 X の異なる二点は, 交わらない近傍を持つ.
				このとき, X の集合のどの有向点族も, 
				X の二つ以上の点に収束すると仮定すると矛盾.
				 (必要条件):
				位相空間 X の集合のどの有向点族も X の高々一つの点に収束するとき, 
				X の任意の異なる二点が互いに交わらない近傍を持つと言えるので, 
				X はハウスドルフになる.
				\newline
				 よってこの主張は正しい.

				\item
				 If $\cal{F}$ converges to p and X is a Hausdorff space, 
				then no other point of X is a cluster point of $\cal{F}$.
				\newline
				 ($\cal{F}$ が点 p に収束し, X がハウスドルフのとき, 
				Xで収積点は唯一つである.)
				\item[\underline{Proof.}]
				 ハウスドルフ空間 X の異なるニ点は, 交わらない近傍を持つから, 
				いかなる有向点族も異なる二点を収積点として持ち得ない. 
				$\cal{F}$ が点 p に収束するとき, これは収積点になり, 唯一つに定まる.\qed
			\end{enumerate}
	\end{description}
\end{document}
